\chapter{\textit{Because}}\label{sec:18}

\section{Introduction}\label{sec:18.1}

In this chapter we explore the meaning of the conjunction \textit{because} by asking what contribution it makes to the meaning of a sentence. \textit{Because} is used to connect two propositions, so its contribution to the meaning of the sentence will be found in the semantic relationship between those two propositions.



We begin in \sectref{sec:18.2} by comparing reason clauses introduced by \textit{because} with time clauses introduced by \textit{when}. Time clauses function as adverbial modifiers, but we will argue that \textit{because} has a different function: it combines two propositions into a new proposition which asserts that a causal relationship exists. An important piece of evidence for this analysis comes from certain scope ambiguities which arise in \textit{because} clauses but not in time clauses.



Conjunctions are often polysemous,\footnote{\citet{Aikhenvald2009}.} and various authors have noted that \textit{because} can have several different semantic functions. Nevertheless, as we discuss in \sectref{sec:18.3}, it has been argued that these different functions do not represent distinct senses. Rather, it is suggested that \textit{because} has just one sense which can be used in different domains of meaning, e.g. propositional meaning vs. utterance meaning. The term \textsc{pragmatic ambiguity} has been proposed to describe such cases.


In \sectref{sec:18.5} we discuss two different causal conjunctions in \ili{German}. One of these introduces subordinate reason clauses and contributes at-issue, truth-conditional content. The other introduces “paratactic” reason clauses and contributes use-conditional content. In \sectref{sec:18.4} we argue that a similar situation holds in English. The conjunction \textit{because} can introduce two different types of clauses: subordinate vs. paratactic. The former contributes at-issue, truth-conditional content, while the latter contributes use-conditional content.


\section{\textit{Because} as a two-place operator}\label{sec:18.2}

Adverbial clauses occur in complex sentences, in which two (or more) propositions are combined to produce a single complex proposition. However, not all adverbial clauses have the same semantic properties. The examples in (\ref{ex:18.1}--\ref{ex:18.2}) illustrate some of the differences between time clauses and reason clauses:


\ea \label{ex:18.1}
\ea  Prince Harry wore his medals when he visited the Pope.\\
\ex Prince Harry didn’t wear his medals when he visited the Pope.\\
\ex Did Prince Harry wear his medals when he visited the Pope?
                       \z
\z

\ea \label{ex:18.2}
\ea  Arthur married Susan because she is rich.\\
\ex Arthur didn’t marry Susan because she is rich.\\
\ex Did Arthur marry Susan because she is rich?
                       \z
\z


All three sentences in \REF{ex:18.1} imply that Harry visited the Pope. As we noted in \chapref{sec:3}, time clauses trigger a presupposition that the proposition they contain is true. Reason clauses do not trigger this kind of presupposition. While sentence (\ref{ex:18.2}a) implies that Susan is rich, sentences (\ref{ex:18.2}b--c) do not carry this inference. Sentence (\ref{ex:18.2}b) could be spoken appropriately by a person who does not believe that Susan is rich, and sentence (\ref{ex:18.2}c) could be spoken appropriately by a person who does not know whether Susan is rich.



So \textit{q because p} does not presuppose that \textit{p} is true; but it entails that both \textit{p} and \textit{q} are true. This entailment is demonstrated in \REF{ex:18.3}.


\ea \label{ex:18.3}
\ea  George VI became King of England because Edward VIII abdicated;\\
  \#but George did not become king.\\
\ex George VI became King of England because Edward VIII abdicated;\\
  \#but Edward did not abdicate.
                       \z
\z


A second difference between time clauses and reason clauses involves the effect of negation. The negative statement in (\ref{ex:18.2}b) is ambiguous. It can either mean ‘Arthur didn’t marry Susan, and his reason for not marrying her was because she is rich;’ or ‘Arthur did marry Susan, but his reason for marrying her was not because she is rich.’ No such ambiguity arises in sentence (\ref{ex:18.1}b).



The time clause in (\ref{ex:18.1}a) functions as a modifier; it makes the proposition expressed in the main clause more specific or precise, by restricting its time reference. \textit{Because} clauses seem to have a different kind of semantic function. \citet{Johnston1994} argues that \textit{because} is best analyzed as an operator CAUSE, which combines two propositions into a single proposition by asserting a causal relationship between the two.\footnote{This operator is probably different from the causal operator involved in morphological causatives, which is often thought of as a relation between an individual and an event/situation.} We might define this operator as shown in \REF{ex:18.4}:


\ea \label{ex:18.4}
\textit{CAUSE(p,q)} is true iff \textit{p} is true, \textit{q} is true, and \textit{p} being true causes \textit{q} to be true.
\z


For example, if \textit{p} and \textit{q} are descriptions of events in the past, \textit{CAUSE(p,q)} would mean that \textit{p} happening caused \textit{q} to happen. A truth table for \textit{CAUSE} would look very much like the truth table for \textit{and}; but there is a crucial additional element of meaning that would not show up in the truth table, namely the causal relationship between the two propositions.\footnote{The definition of causality is a long-standing problem in philosophy, which we will not address here. One way to think about it makes use of a counter-factual conditional (see \chapref{sec:19}): \textit{CAUSE(p,q)} means that if \textit{p} had not happened, \textit{q} would not have happened either.}



This analysis provides an immediate explanation for the ambiguity of sentence (\ref{ex:18.2}b) in terms of the scope of negation:


\ea \label{ex:18.5}
  \textit{Arthur didn’t marry Susan because she is rich.}\\
\ea  ¬CAUSE(RICH(s), MARRY(a,s))\\
\ex CAUSE(RICH(s), ¬MARRY(a,s))
                       \z
\z


If this approach is on the right track, we would expect to find other kinds of scope ambiguities involving \textit{because} clauses as well. This prediction turns out to be correct: in sentences of the form \textit{q because p}, if the first clause contains a scope-bearing expression such as a quantifier, modal, or propositional attitude verb, that expression may be interpreted as taking scope either over the entire sentence or just over its immediate clause. Some examples are provided in (\ref{ex:18.6}--\ref{ex:18.7}).


\ea \label{ex:18.6}
\textit{Few people admired Churchill because he joined the Amalgamated Union of Building Trade Workers.}\\
\ea  CAUSE(JOIN(c,aubtw), [few x: person(x)] ADMIRE(x,c))\\
\ex{} [few x: person(x)] CAUSE(JOIN(c,aubtw), ADMIRE(x,c))
                       \z
\z

\ea \label{ex:18.7}
\textit{I believed that you love me because I am gullible.}\\
\ea  BELIEVE(s, CAUSE(GULLIBLE(s), LOVE(h,s))\\
\ex CAUSE(GULLIBLE(s), BELIEVE(s, LOVE(h,s)))\\
    {}[s = speaker; h = hearer]
                       \z
\z


One reading for sentence \REF{ex:18.6}, which is clearly false in our world, is that only a few people admired Churchill, and the reason for this was that he joined the AUBTW. The other reading for sentence \REF{ex:18.6}, very likely true in our world, is that only a few people’s admiration of Churchill was motivated by his joining of the AUBTW; but many others may have admired him for other reasons. (The reader should work out the two readings for sentence \REF{ex:18.7}.)


\section{Semantic functions of  \textit{because}}\label{sec:18.3}

Now let us consider the apparent polysemy of \textit{because}. \citet[76--78]{Sweetser1990} suggests that \textit{because} (and a number of other conjunctions) can be used in three different ways: 


\begin{quote}
Conjunction may be interpreted as applying in one of (at least) three domains [where] the choice of a “correct” interpretation depends not on form, but on a pragmatically motivated choice between viewing the conjoined clauses as representing content units, logical entities, or speech acts.\newline [\citeyear[78]{Sweetser1990}]
\end{quote}

\ea \label{ex:18.8}
\ea  John came back because he loved her.  \hfill  [\textsc{content} domain]\\
\ex John loved her, because he came back. \hfill   [\textsc{epistemic} domain]\\
\ex What are you doing tonight, because there’s a good movie on. \\
\hfill [\textsc{speech act} domain]
                       \z
\z


The content domain has to do with real-world causality; in (\ref{ex:18.8}a), John’s love causes him to return. The epistemic domain (\ref{ex:18.8}b) has to do with the speaker’s grounds for making the assertion expressed in the main clause: the content of the \textit{because} clause (\textit{he came back}) provides evidence for believing the assertion (\textit{John loved her}) to be true. Many authors refer to this use as the \textsc{evidential} use of because.


Sweetser explains the speech act domain (\ref{ex:18.8}c) as follows:


\begin{quote}
{}[T]he \textit{because} clause gives the cause of the \textit{speech act} embodied by the main clause. The reading is something like ‘I \textit{ask} what you are doing tonight because I want to suggest that we go see this good movie.’ [\citeyear[77]{Sweetser1990}]
\end{quote}


We might (somewhat informally) express the causal relations involved in these three types of reason clause as shown in (\ref{ex:18.40}a–c).


\ea \label{ex:18.40}
\ea  \textit{John came back because he loved her.}  \hfill  [real-world causation]
CAUSE(LOVE(j,m), COME\_BACK(j))  \hfill (assuming \textit{her} = Mary)\\
\ex \textit{John loved her, because he came back.} \hfill   [evidential use] 
CAUSE(COME\_BACK(j), I have grounds to assert that LOVE(j,m))\\
\ex \textit{What are you doing tonight, because there’s a good movie on.}
\newline
CAUSE(there’s a good movie on, I ask you what you are doing tonight) 
\hfill [speech act use]
                       \z
\z


Sweetser denies that the three uses above involve different senses of \textit{because}. Rather, she argues that \textit{because} has a single sense which can operate on three different levels, or domains, of meaning. She describes this situation, taking a term from \citet{Horn1985}, as a case of \textsc{pragmatic ambiguity}; in other words, an ambiguity of usage rather than an ambiguity of sense.



This seems like a very plausible suggestion, but any such proposal needs to account for the fact that the various uses of \textit{because} are distinguished by a number of real differences, both semantic and structural. The most obvious of these is the presence of pause, or “comma intonation”, between the two clauses. The pause appears to be optional with “content domain” uses of \textit{because}, as in (\ref{ex:18.9}a), but obligatory with other uses. If the pause is omitted in (\ref{ex:18.9}b--c), the sentences can only be interpreted as expressing real-world causality, even though this interpretation is somewhat bizarre. (With the pause, (\ref{ex:18.9}b) illustrates an evidential use while (\ref{ex:18.9}c) illustrates a speech act use.)


\ea \label{ex:18.9}
\ea  Mary scolded her husband (,) because he forgot their anniversary.\\
\ex Arnold must have sold his Jaguar \#(,) because I saw him driving a 1995 minivan.\\
\ex Are you hungry \#(,) because there is some pizza in the fridge?
                       \z
\z


We will suggest in \sectref{sec:18.4} that the presence vs. absence of pause or intonation break in these sentences is an indicator of different syntactic structures (subordinate vs. paratactic), which in turn determine the range of potential semantic functions. But first we will consider different types of reason clauses in German, a language in which syntactic subordination is clearly marked.


\section{Two words for ‘because’ in German}\label{sec:18.5}


\ili{German} has two different words which are translated as ‘because’.\footnote{A number of other languages also have two words for ‘because’, including Modern \ili{Greek}, \ili{Dutch}, and \ili{French} (\citealt{Pit2003}; \citealt{Kitis2006}).} Both of these words can be used to describe real-world causality, as illustrated in (\ref{ex:18.27}--\ref{ex:18.28}). In each case, the a and b sentences have the same English translation.\footnote{The material in this section is based almost entirely on the work of Tatjana \citet{Scheffler2005,Scheffler2008}, and all examples that are not otherwise attributed come from these works.} 


\ea \label{ex:18.27}
\ea  \gll Ich  habe  den  Bus  verpasst,  \textit{weil}  ich  spät  dran  war.\\
\textsc{1sg}  \textsc{aux}  the.\textsc{acc}  bus  missed  because  \textsc{1sg}  late  there  was\\
\glt ‘I missed the bus because I got there late.’ \\
\ex  Ich habe den Bus verpasst, \textit{denn} ich war spät dran. (same meaning)\footnote{\url{http://answers.yahoo.com}}
\z \z

\ea \label{ex:18.28}
\ea  \gll  Die  Straße  ist  ganz  naß,  \textit{weil}  es  geregnet  hat.\\
the.\textsc{nom}  street  is  all  wet  because  it  rained  \textsc{aux}\\
\glt ‘The street is wet because it rained.’ \\
\ex Die Straße ist ganz naß, \textit{denn} es hat geregnet. (same meaning)\footnote{\citet[§3.1]{Scheffler2008}}
\z \z

However, in other contexts the two words are not interchangeable. As demonstrated by \citet{Scheffler2005,Scheffler2008}, only \textit{denn} can be for translating reason clauses in Sweetser’s evidential \REF{ex:18.29} and speech act \REF{ex:18.30} domains; \textit{weil} would be impossible in these contexts.


\ea \label{ex:18.29}
\ea  \gll  Es hat geregnet, \textit{denn} die Straße ist ganz naß.\\
it  has rained because the.\textsc{nom}  street  is  all  wet \\
\glt ‘It was raining, because the street is wet.’ \\
\ex *Es hat geregnet, \textit{weil} die Straße ganz naß ist.
                       \z
\z

\ea \label{ex:18.30}
\ea  \gll Ist  vom  Mittag  noch  etwas  übrig?\\
is  from  midday  still  anything  left.over\\
\gll \textit{Denn}  ich  habe  schon  wieder  Hunger.\\
because  \textsc{1sg}  have  already  again  hunger\\
\glt ‘Is there anything left over from lunch? Because I’m already hungry again.’ \\
\ex ?? Ist vom Mittag noch etwas übrig? \textit{Weil} ich schon wieder Hunger habe.
\z \z


With respect to the contrast discussed in \chapref{sec:11} between truth-conditional (at-issue) vs. use-conditional meaning, Scheffler shows that \textit{weil} introduces at-issue reason clauses, whereas \textit{denn} introduces use-conditional reason clauses. We briefly summarize here some of the evidence that supports this claim. First, as illustrated in \REF{ex:18.34}, \textit{weil} clauses can be interpreted within the scope of main clause negation, whereas \textit{denn} clauses cannot.


\ea \label{ex:18.34}
\ea  \gll  Paul  ist  nicht  zu  spät  gekommen,  \textit{weil}  er  den  Bus  verpaßt  hat.\\
Paul  \textsc{aux}  \textsc{neg}  too  late  come  because  he  the.\textsc{acc}  bus  missed  \textsc{aux}\\
\gll   {}[Sondern  er  hatte  noch  zu  tun.]\\
  rather  he  had  still  to  do\\
  \glt ‘Paul wasn’t late because he missed the bus. [But rather, because he still had work to do.]’ \\
\ex  \#Paul ist nicht zu spät gekommen, \textit{denn} er hat den Bus verpaßt. [Sondern er hatte noch zu tun.]
\z \z


Similarly, \textit{weil} clauses in questions can be interpreted as part of what is being questioned, that is, within the scope of the interrogative force (\ref{ex:18.35}a). \textit{Denn} clauses cannot be interpreted in this way, as shown in (\ref{ex:18.35}b).


\ea \label{ex:18.35}
\ea \gll  Wer kam zu spät, \textit{weil} er den Bus verpaßt hat?\\
who  came  too  late  because  he the bus missed has \\
\glt ‘Who was late because he missed the bus?’ \\
\ex ?? Wer kam zu spät, \textit{denn} er hat den Bus verpaßt?
                       \z
\z


\textit{Weil} clauses can be used to answer a \textit{why} question, in which the causal relation is the focus of the question (\ref{ex:18.33}a). \textit{Denn} clauses cannot be used in this way (\ref{ex:18.33}b).


\ea \label{ex:18.33}
\ea  \gll Warum  ist  die  Katze  gesprungen? \\
why  \textsc{aux}  the.\textsc{nom}  cat  jumped \\
\newline
\gll — \textit{Weil}  sie  eine  Maus  sah. \\
 —  because  she  a  mouse  saw\\
\glt ‘Why did the cat jump? — Because it saw a mouse.’ \\
\ex  —*\textit{Denn} sie sah eine Maus.
\z \z



\textit{Denn} clauses cannot be embedded within a subordinate clause, whereas this is possible with \textit{weil} clauses. Example \REF{ex:18.36} illustrates this contrast in a complement clause, and \REF{ex:18.37} in a conditional clause.


\ea \label{ex:18.36}
\ea   \gll Ich  glaube  nicht,  daß  Peter  nach  Hause  geht,  \textit{weil}  er  Kopfschmerzen  hat.\\
\textsc{1sg}  believe  \textsc{neg}  \textsc{comp}  Peter  to  home  goes  because  he  headache  has\\
\glt ‘I don’t believe that Peter is going home because he has a headache.’ \\
\ex  \#Ich glaube nicht, daß Peter nach Hause geht, \textit{denn} er hat Kopfschmerzen.\footnote{This sentence cannot be used with the intended meaning, in which the speaker expresses disbelief that the headache was the reason for Peter’s going home. The \textit{denn} clause would be grammatical as a parenthetical comment, but in this particular example that parenthetical reading seems semantically incoherent.}
\z \z

\ea \label{ex:18.37}
\ea  Wenn Peter zu spät kam, \textit{weil} er den Bus verpaßt hat, war es seine eigene Schuld.\\
\glt ‘If Peter was late because he missed the bus, it was his own fault.’ \\
\ex \#Wenn Peter zu spät kam, \textit{denn} er hat den Bus verpaßt, war es seine eigene Schuld.
\z \z


Cross-linguistically, it appears that at-issue reason clauses always express real-world causation. Use-conditional reason clauses, in contrast, can typically express any of the three semantic functions identified by Sweetser. The differences between German \textit{weil} vs. \textit{denn} provide a very clear instance of this pattern.


In addition to differences of semantic type and function, there are structural differences between the two conjunctions as well: \textit{weil} is a subordinating conjunction, whereas \textit{denn} is not. I will refer to the structure of \textit{denn} clauses as \textsc{paratactic}.\footnote{Paratactic structure is a characteristic feature of parenthetical comments. In adopting this analysis I diverge from Scheffler, who describes the structure of \textit{denn} clauses as a special type of co-ordination.} The difference between subordinate vs. paratactic or co-ordinate structure in German is clearly visible due to differences in word order. In German main clauses, the auxiliary verb (or tensed main verb if there is no auxiliary) occupies the second position in the clause, as illustrated in (\ref{ex:18.31}a). This is also the order observed in paratactic and co-ordinate clauses. In subordinate clauses, however, the auxiliary or tensed main verb occupies the final position in the clause, as illustrated in (\ref{ex:18.31}b).\footnote{This is true for subordinate clauses which are introduced by a conjunction or complementizer. Where there is no conjunction or complementizer at the beginning of the subordinate clause, the auxiliary or tensed main verb occupies the second position.}


\ea \label{ex:18.31}
\ea   \gll Ich  \textit{habe}  dieses  Buch  gelesen.\\
\textsc{1sg}.\textsc{nom}  have  this.\textsc{acc}  book  read.\textsc{ptcp}\\
\glt ‘I have read this book.’
\ex \gll Sie  sagt,  daß  er  dieses  Buch  gelesen  \textit{hätte}.\\
\textsc{3sg}.\textsc{f}.\textsc{nom}  says  that  \textsc{3sg}.\textsc{m}.\textsc{nom}  this.\textsc{acc}  book  read  have.\textsc{sbjv}\\
\glt ‘She says that he has read this book.’
\z \z


Looking back at examples (\ref{ex:18.27}--\ref{ex:18.28}), we can see that the tensed verbs \textit{war} ‘was’ and \textit{hat} ‘has’ occur in second position following \textit{denn} but in final position following \textit{weil}. This contrast provides a clear indication that \textit{weil} clauses are subordinate while \textit{denn} clauses are not. Further evidence comes from the fact that \textit{weil} clauses can be fronted but \textit{denn} clauses cannot, as shown in \REF{ex:18.32}. 

\largerpage
\ea \label{ex:18.32}
\ea  \textit{Weil} es geregnet hat, ist die Straße naß.\\
\glt ‘Because it rained, the street is wet.’
\ex *\textit{Denn} es hat geregnet, ist die Straße naß.\\
\z \z


The pattern that we observe in \ili{German} seems to hold, at least as a fairly strong tendency, across languages: paratactic reason clauses (like those marked with \textit{denn} in \ili{German}) always contribute use-conditional meanings. There also seems to be a fairly strong preference across languages for subordinate reason clauses to be interpreted as contributing at-issue meaning.\footnote{Some evidence suggests that this second correlation may turn out to be a kind of pragmatic inference, perhaps a Generalized Conversational Implicature. Alternatively, we might analyze the occasional evidential or speech act uses of subordinate reason clauses as coercion effects.}



\section{Subordinate vs. paratactic \textit{because} in English}\label{sec:18.4}


English \textit{because} has both subordinating and paratactic uses. The two uses are distinguished by the comma intonation, which is strongly preferred (if not obligatory) in the paratactic structure but not possible with true subordinate structure.\footnote{Paratactic \textit{because} seems to be semantically equivalent to causal \textit{for}, and has largely replaced causal \textit{for} in modern spoken English.}  Paratactic \textit{because} can be used to express all three of the semantic functions discussed in \sectref{sec:18.3}: real-world causation (\ref{ex:18.41}a), evidential (\ref{ex:18.41}b), and speech act (\ref{ex:18.41}c).


\ea \label{ex:18.41}
\ea  Mary scolded her husband, because he forgot their anniversary.
\ex Arnold must have sold his Jaguar, because I saw him driving a 1995 minivan.
\ex The last train leaves at 6 PM, because you must be anxious to get home.
                       \z
\z


Subordinating \textit{because} allows only the first of these (\ref{ex:18.42}a). Even in contexts where the evidential (\ref{ex:18.42}b) or speech act (\ref{ex:18.42}c) readings would be pragmatically much more plausible, when read as a single intonation unit only the odd, real-world causation reading is possible.


\ea \label{ex:18.42}\judgewidth{??}
\ea[]{Mary scolded her husband because he forgot their anniversary.}
\ex[??]{Arnold must have sold his Jaguar because I saw him driving a 1995 minivan.}
\ex[??]{The last train leaves at 6 PM because you must be anxious to get home.}
 \z
\z


Only subordinating \textit{because} clauses can be fronted. For this reason, fronted \textit{because} clauses allow only the real-world causation reading:


\ea \label{ex:18.15}\judgewidth{??}
\ea[]{Because it’s raining, we can’t go to the beach. \hfill [\textsc{content}]}
\ex[??]{Because I saw Arnold driving a 1995 minivan, he sold his Jaguar. \\ \hfill [*\textsc{evidential}]}
\ex[??]{Because you must be anxious to get home, the last train leaves at 6 PM.\footnote{These judgments are much stronger with causal \textit{for}, which is always paratactic: *\textit{The last train leaves at 6 PM for you must be anxious to get home}; *\textit{For you must be anxious to get home, the last train leaves at 6 PM}.} \hfill [*\textsc{speech act}]}
                       \z
\z


Consistent with the pattern described in the previous section, subordinating \textit{because} clauses express at-issue content, whereas paratactic \textit{because} clauses express use-conditional meaning. For this reason, subordinating \textit{because} can be negated, questioned, and embedded within a conditional, whereas none of these things are possible with paratactic \textit{because}.


Since subordinating \textit{because} can be negated, examples like (\ref{ex:18.43}a) involve a scope ambiguity. In this case the more plausible reading (the husband was scolded, not for forgetfulness but for some other reason) requires negation to take scope over causation. This reading is not available in the paratactic example (\ref{ex:18.43}b). The only possible reading for that example is the one in which causation takes scope over negation: because he forgot, he was not scolded (a highly unlikely scenario).


\ea \label{ex:18.43}
\ea  Mary didn’t scold her husband because he forgot their anniversary.
\ex Mary didn’t scold her husband, because he forgot their anniversary.
                       \z
\z


Example (\ref{ex:18.44}a) illustrates how the causal relation itself can be questioned with subordinating \textit{because}. This kind of question is not possible with paratactic \textit{because}; in (\ref{ex:18.44}b) the reason clause would normally be interpreted as the speaker’s reason for asking (speech act function).


\ea \label{ex:18.44}
\ea  Did Mary scold her husband because he forgot their anniversary?
\ex Did Mary scold her husband? because he forgot their anniversary.
                       \z
\z


In questions, as in statements, when the speech act function is intended, only paratactic \textit{because} is possible (\ref{ex:18.45}a, \ref{ex:18.46}a). Subordinating \textit{because} gives rise to bizarre readings in which real-world causation is being questioned, as in (\ref{ex:18.45}b) and (\ref{ex:18.46}b).


\ea \label{ex:18.45}
\ea  Did Arnold sell his Jaguar? because I just saw him driving a 1995 minivan.
\ex Did Arnold sell his Jaguar because I just saw him driving a 1995 minivan?
                       \z
\ex \label{ex:18.46}
\ea  Are you hungry? because there is some pizza in the fridge.
\ex Are you hungry because there is some pizza in the fridge?
                       \z
\z


Example (\ref{ex:18.12}a) illustrates how subordinating \textit{because} can be embedded within a conditional, with the causal relation itself understood to be part of the antecedent. However, as our analysis predicts, this is only possible with the real-world causation reading. Example (\ref{ex:18.47}a) is odd because it seems to involve a subordinate structure, but pragmatic considerations strongly favor the evidential reading. Paratactic \textit{because} in a conditional construction, as in (\ref{ex:18.12}b, \ref{ex:18.47}b), can only be interpreted as a parenthetical side comment, not part of either the antecedent or consequent.


\ea \label{ex:18.12}
\ea[]{If Mary scolded her husband because he forgot their anniversary, they will be back on speaking terms in a few days.}
\ex[]{If Mary scolded her husband, because he forgot their anniversary, they will be back on speaking terms in a few days.}
\z 
\ex \label{ex:18.47}
\ea[??]{If Arnold sold his Jaguar because I just saw him driving a 1995 minivan, he is  likely to regret it.}
\ex[]{If Arnold sold his Jaguar, because I just saw him driving a 1995 minivan, he is  likely to regret it.}
\z \z


To summarize, we have proposed that adverbial clauses introduced by \textit{because} can occur in two different structural configurations, paratactic or subordinate. Paratactic \textit{because} clauses must be separated from the main clause by a pause or intonation boundary, but such a boundary is not allowed before subordinate \textit{because} clauses (when they follow the main clause). The paratactic structure contributes use-conditional meaning, and is compatible with any of Sweetser’s three functions of \textit{because}. Subordinate \textit{because} clauses contribute to the at-issue (truth-conditional) meaning of the sentence, and always describe real-world causation.





\section{Conclusion}\label{sec:18.6}

In this chapter we have distinguished two types of reason clauses: truth-conditional vs. use-conditional. These two types can be distinguished using familiar tests for truth-conditional (at-issue) content. First, truth-conditional reason clauses can be part of what is negated or questioned when the sentence as a whole is negated or questioned, but this is not the case with the use-conditional type. Second, truth-conditional reason clauses can be embedded within the antecedent clause of a conditional construction, but use-conditional reason clauses cannot.\footnote{Challengeability  was not discussed in this chapter, but this test can also be used to confirm the distinction we are making here. The truth of a statement can be appropriately challenged based on the causal relationship expressed in a truth-conditional reason clause, but not on that expressed in a use-conditional reason clause.}


We have shown that use-conditional reason clauses can be used to express a variety of causal relations. The three most prominent of these are (a) real-world causation; (b) the evidential basis for asserting the content of the main clause; or (c) the reason for performing the speech act contained in the main clause. Truth-conditional reason clauses, on the other hand, always express real-world causation.  


We have also identified two different structural configurations which may be used to express reason clauses: subordinate vs. paratactic. In \ili{German} the two structures are introduced by different conjunctions: \textit{weil} for subordinate reason clauses, and \textit{denn} for paratactic reason clauses. The English causal conjunction \textit{because} may occur in either structure. Intonation provides one criterion for distinguishing the two uses: paratactic \textit{because} clauses are normally separated from the main clause by a pause (or “comma intonation”), but this pause is not allowed before subordinate \textit{because} clauses. Some other diagnostics for distinguishing the two structures, both in English and in \ili{German}, include the following: (i)~Subordinate reason clauses can be fronted, but paratactic reason clauses cannot. (ii)~Scope ambiguities involving negation, quantifiers, modals, or propositional attitude verbs are possible with subordinate reason clauses, but not with paratactic reason clauses.


We have argued for both English and \ili{German} that subordinate reason clauses always express truth-conditional meaning, whereas paratactic reason clauses express use-conditional meaning. The same generalization seems to hold for a number of other European languages as well. (The rare cases in which subordinate reason clauses express a use-conditional function should perhaps be analyzed as arising through a type of coercion.) Additional research is required to see whether this generalization holds with equal strength among non-European languages. 



\furtherreading{ 
\citet{Sæbø1991} and (\citeyear{Sæbø2011}: §3.3) provide a good overview of the semantics of causal connectives like \textit{because}, and a comparison with other types of adverbial connectives. \citet{Lewis1973b} and (\citeyear{Lewis2000}) lay out two different versions of his counterfactual analysis of causation. \citet[ch. 4]{Scheffler2013} provides a detailed discussion of the syntax and semantics of the two \ili{German} conjunctions meaning ‘because’.
}

\discussionexercises{

\paragraph*{A: Explain the scopal ambiguity of the following sentences, and state the two readings in logical notation: \\}

\begin{enumerate}
\item  Arthur didn’t marry Susan because she is rich.
\newpage ~ 
\smallskip 
\vspace*{-4mm}

\shortmodelanswer{Model answer}{
\begin{enumerate}[label=\alph*.]
\item  ¬CAUSE(RICH(s), MARRY(a,s))
\item  CAUSE(RICH(s), ¬MARRY(a,s))
\end{enumerate}
}
\item  Mrs. Thatcher will not win because she is a woman. \\
(spoken in 1979)
\item  Tourists rarely visit Delhi because the food is so spicy.
\item  I doubt that Peter is happy because he was fired.
\end{enumerate}
 

\paragraph*{B: Show how you could use some of the tests discussed in this chapter to determine whether the \textit{because} clauses in the following examples contribute truth-conditional or use-conditional meaning:}

\begin{enumerate}
\item  Arthur works for the State Department, because he has a STATE.GOV e-mail address.
\item  Oil prices are rising, because OPEC has agreed to cut production.
\end{enumerate}
}
\homeworkexercises{
In \sectref{sec:18.2} we proposed the following analysis for the scopal ambiguity of sentence (\ref{ex:18.2}b):
\textit{Arthur didn’t marry Susan because she is rich.}
\begin{itemize}[noitemsep]
\item[i.]  ¬CAUSE(RICH(s), MARRY(a,s))\\
\item[ii.]  CAUSE(RICH(s), ¬MARRY(a,s))
\end{itemize}

Provide a similar analysis showing the two possible readings for each of the following sentences. If you wish, you may write out the clauses in prose rather than using formal logic notation, e.g.: ¬CAUSE(\textit{Susan is rich}, \textit{Arthur marry Susan}).

\begin{enumerate}
\item Steve Jobs didn’t start Apple because he loved technology.\footnote{\url{https://www.fastcompany.com/3001441/do-steve-jobs-did-dont-follow-your-passion}}
\item Arnold must have sold his Jaguar because I saw him driving a minivan.
\item Few Texans voted for Romney because he is a Mormon.
\item Susan believes that A.G. Bell was rich because he invented the telephone.
\end{enumerate}
}
